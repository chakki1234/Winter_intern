\renewcommand{\theequation}{\theenumi}
\begin{enumerate}[label=\thesection.\arabic*.,ref=\thesection.\theenumi]
\numberwithin{equation}{enumi}

\item aigiri.txt contains a slokam in Telugu it is read and all the characters are stored to the variable $sloka\_txt$ with the help of the code below.
\begin{align*}
sloka &= open('aigiri.txt', 'r') \\
sloka\_txt &= sloka.read() \\
sloka &. close()
\end{align*}

\item $sloka\_txt$ is a string containing all the words it is split into individual words and the list of words are saved to the variable $words$. A new file $unicode.txt$ is opened to write the converted unicode.
\begin{align*} 
words &= sloka\_txt.split() \\
uni_file &= open('unicode.txt', 'w')
\end{align*}

\item A  for loop runs through all the words in the list and checks if the word is $'|'$ or a number. If so it does not convert the word to its unicode value and proceeds with the next word. If the word is neither $'|'$ nor a number, an other for loop is used to access each character of the word. Each character is then passed on to a function called $ord()$ which returns the integer code point value of the character. The integer value is then converted to its hexadecimal value using the function $hex()$. The hexadecimal string contains the character $'x'$ which is replaced with a null character to get the actual hexadecimal value it is then appedned to the string $'U+'$ and the resultant is written onto the txt file. The process is repeated for the remaining characters in the word and for all the remaining words.
\begin{align*}
\ \ \ \ &for\ i\  in\  words:\\
&if(i !='|'\  and\  not(i.isdigit())):\\
&for\  j\  in\  i:\\
&uni\_file.write('U+'\  +\  hex(ord(j)).replace('x', '\ '))\\
&uni\_file.write('\ \ ')\\
&uni\_file.close()
\end{align*} 

\end{enumerate}
